%-------------------------------------------------------------------------------
% Proposal Skripsi
%-------------------------------------------------------------------------------

%Template pembuatan naskah skripsi.
\documentclass{tif-uin-suka}

%Untuk prefiks pada daftar gambar dan tabel
\usepackage[titles]{tocloft}
\renewcommand\cftfigpresnum{Gambar\  }
\renewcommand\cfttabpresnum{Tabel\   }

%Untuk hyperlink dan table of content
\usepackage{hyperref}
\newlength{\mylenf}
\settowidth{\mylenf}{\cftfigpresnum}
\setlength{\cftfignumwidth}{\dimexpr\mylenf+2em}
\setlength{\cfttabnumwidth}{\dimexpr\mylenf+2em}

%Untuk Bold Face pada Keterangan Gambar
\usepackage[labelfont=bf]{caption}

%Untuk caption dan subcaption
\usepackage{caption}
\usepackage{subcaption}
\usepackage{geometry}

\usepackage{index}
\def\Index#1{\def\x##1##2{\MakeUppercase{##1}{##2}}\textit{\x#1} \index{\x#1}} 


%-----------------------------------------------------------------
%Disini awal masukan untuk data proposal skripsi
%-----------------------------------------------------------------
\titleind{Rancang Bangun Aplikasi Berita Berbasis Mobile Dengan Metode Test Driven Development}

\fullname{Aji Kisworo Mukti}

\idnum{10651067}

\approvaldate{6 September 2017}

\def \createddate{6 September 2017}

\degree{Sarjana Komputer}

\yearsubmit{2017}

\program{Teknik Informatika}

\dept{Teknik Informatika}

\firstsupervisor{Aulia Faqih Rifa'i}
\firstnip{19860306 201101 1 009}

\secondsupervisor{Nama Dosen 2}
\secondnip{NIP Dosen 2}

\def \approvalnumber{FM-UINSK-BM-06-03/R0}
\def \shortuniversity{UIN Sunan Kalijaga}

% Term untuk penulis, apakah "penulis", "penyusun" atau "peneliti" silakan ganti
\def \writerlabel{peneliti}
\def \writerlabelCap{Peneliti}

\def \city{Yogyakarta}


%-----------------------------------------------------------------
%Disini akhir masukan untuk data proposal skripsi
%-----------------------------------------------------------------

\begin{document}

\coverProposal

%-----------------------------------------------------------------

%-----------------------------------------------------------------
%Disini awal masukan untuk Bab
%-----------------------------------------------------------------
%!TEX root = ../skripsi.tex
%-------------------------------------------------------------------------------
% 								BAB I
% 							LATAR BELAKANG
%-------------------------------------------------------------------------------
\begin{spacing}{2}
\chapter{LATAR BELAKANG}

\section{Latar Belakang Masalah}
Media massa adalah alat yang digunakan dalam penyampaian pesan-pesan dari sumber kepada khalayak (menerima) dengan menggunakan alat-alat komunikasi mekanis seperti surat kabar, film, radio, TV.\cite{Hafied2005}. Pada perkembangannya, media massa mulai menggunakan teknologi informasi sebagai perantara yang memungkinkan publik berinteraksi lebih cepat dalam mengabarkan berita.\cite{maulana2016}. Ini ditandai dengan munculnya berbagai situs berita milik media massa yang dapat diakses menggunakan perangkat komputer maupun \emph{mobile}.

Kebutuhan mengakses berita melalui media perangkat \emph{mobile} meningkat sesuai dengan hasil riset Pew Research Center yang menyatakan bahwa setengah dari pengguna \emph{smartphone} menggunakan perangkatnya untuk mengakses berita pada tahun 2011.\cite{journalism2012}. Hal ini memperlihatkan bahwa pembaca berita berbasis \emph{mobile} (mobile news) mungkin mengikuti berita secara berkala. Selain itu, pembaca berita berbasis \emph{mobile} memiliki pola penggunaan media dan preferensi berita tertentu.\cite{sylvia2012}. Seperti dengan cara mengakses langsung berita tertentu melalui situs atau aplikasi terkait.\cite{journalism2012}.

Melihat perkembangan tersebut, maka sebuah media massa sudah seharusnya memiliki aplikasi berita \emph{mobile} guna melayani kebutuhan pembaca. Berbagai media massa ternama seperti NYTimes, Wall Street Journal atau media massa lokal seperti Kompas, Detik, dan beberapa media massa lain telah mengembangkan aplikasi berita berbasis \emph{mobile}. Namun pada kenyataannya beberapa aplikasi \emph{mobile} yang dimiliki media massa tersebut sering mengalami masalah dalam penggunaannya, seperti masalah yang sering terjadi pada aplikasi \emph{mobile} milik NYTimes.\cite{martin2011}.

Masalah penggunaan yang dimaksud adalah masalah teknis yang sering terjadi dari aplikasi \emph{mobile}. Penyebab masalah teknis ini sering dilakukan pada proses pengembangan, seperti tidak melakukan proses pengujian yang tepat.\cite{shiv2015}.

Proses pengujian dalam pengembangan tidak lepas dari metode pengembangan yang digunakan. Pada aplikasi berbasis \emph{mobile} metode pengembangan dengan sistem dan pendekatan berbasis proses yang intensif (\emph{proccess-intensive}) berubah menjadi lebih menggunakan pendekatan berbasis \emph{agile} atau proses yang lebih lincah. \emph{Test Driven Development} merupakan salah satu metode pengembangan yang masuk dalam kategori \emph{agile} yang paling sering digunakan.\cite{wasserman2010}.

Dalam studi kasus yang dilakukan oleh IBM Corporation dan North Carolina State University ditemukan bahwa kode sumber yang dikembangkan dengan metode \emph{Test Driven Development} menunjukan selama pengujian verifikasi dan urutan (\emph{regression}), 40\% lebih sedikit mengalami masalah daripada aplikasi yang dikembangkan dengan cara yang lebih tradisional.\cite{laurie2003}

Guna dapat mengembangkan aplikasi \emph{mobile} yang lebih sedikit mengalami masalah, maka penulis mengangkat penelitian ini dengan judul "Rancang Bangun Aplikasi Berita Berbasis Mobile Dengan Metode Test Driven Development".

\section{Rumusan Masalah}
Rumusan permasalahan yang akan diselesaikan dalam penelitian ini yaitu:
\begin{enumerate}
  \item Bagaimana merancang bangun aplikasi berita yang nyaman dan stabil untuk pengguna pada perangkat \emph{mobile}.
  \item Bagaimana menerapkan metode \emph{Test Driven Development} dalam merancang dan membangun aplikasi berbasis \emph{mobile}.
\end{enumerate}

\section{Batasan Masalah}
Batasan masalah yang akan dibahas pada penelitian ini sebagai berikut:
\begin{enumerate}
  \item Penelitian ini fokus pada implementasi metode \emph{test driven development}.
  \item Objek yang dijadikan penelitian adalah aplikasi berita.
  \item Teknologi atau sistem yang digunakan untuk mengelola berita tidak dibahas pada penelitian ini.
  \item Aplikasi ini dibuat menggunakan kerangka kerja React Native.
  \item Penelitian ini tidak membahas mengenai bahasa pemrograman, kerangka kerja, dan pustaka yang digunakan dalam pengembangan.
\end{enumerate}


\section{Tujuan Penelitian}
Sesuai dengan masalah yang telah dirumuskan, maka tujuan dari penelitian ini adalah menghasilkan aplikasi berbasis \emph{mobile} yang dapat menyajikan berita menggunakan metode \emph{test driven development}.
\begin{enumerate}
  \item Menghasilkan aplikasi berbasis \emph{mobile} untuk dapat menyajikan berita.
  \item Dapat mengimplementasikan metode \emph{test driven development} kedalam sistem aplikasi berbasis \emph{mobile} yang dirancang.
\end{enumerate}


\section{Manfaat Penelitian}
Manfaat penelitian yang diharapkan dapat menyediakan Aplikasi Berita \emph{Mobile} untuk menyajikan berita kepada pengguna, guna kemudahan dalam mengakses informasi.

\end{spacing}
% Baris ini digunakan untuk membantu dalam melakukan sitasi
% Karena diapit dengan comment, maka baris ini akan diabaikan
% oleh compiler LaTeX.
\begin{comment}
\bibliography{daftar-pustaka}
\end{comment}

%!TEX root = ../skripsi.tex
%-------------------------------------------------------------------------------
%                            BAB II
%               TINJAUAN PUSTAKA DAN LANDASAN TEORI
%-------------------------------------------------------------------------------
\begin{spacing}{2}
\chapter{TINJAUAN PUSTAKA DAN LANDASAN TEORI}                

\section{Tinjauan Pustaka}
  Berdasarkan penelitian Pew Research Center (2012) yang berjudul "\emph{Mobile Devices and News Consumption: Some Good Signs for Journalism}" menyatakan bahwa pembaca menghabiskan waktu lebih lama di aplikasi berita pada perangkat \emph{mobile}, mengunjungi lebih banyak halaman, dan frekuensi kembali mengunjungi lebih banyak dibandikan pada perangkat komputer biasa.\cite{journalism2012}. Pernyataan tersebut diperkuat oleh penelitian Sylvia Chan-Olmsted, Hyejoon Rim1, dan Amy Zerba yang menyimpulkan bahwa faktor-faktor yang mempengaruhi waktu adopsi pengguna mungkin berbeda dari yang mempengaruhi tingkat penggunaan atau kemauannya untuk membayar konten. Gagasan ini terutama berlaku untuk konsumsi berita melalui platform media "hibrida" seperti berita mobile.\cite{sylvia2012}. Dengan demikian sudah layak untuk diperhitungkan jika sebuah media berita harus memiliki sebuah aplikasi berita untuk pembaca. Namun, untuk memanfaatkan kesempatan itu sebuah media harus mengerjakannya lebih baik dibandingkan pada lingkungan \emph{desktop} dalam kecepatan memahami perilaku pengguna dan pengembangan teknologi.\cite{journalism2012}.

  Pada sisi pengembangan teknologi tidak lepas dari metode yang digunakan dalam pengembangan aplikasi berbasis mobile. Seperti pada penelitian Novri Asyara Mahyudanil yang berjudul "Rancang Bangun \emph{Mobile Commerce} Berbasis Android \emph{Smartphone} Dengan \emph{Extreme Programming}". Penelitian tersebut menggunakan metode pengembangan \emph{extreme programming} karena mengutamakan keterlibatan pengguna dalam mengembangkan perangkat lunak dan lebih mengedepankan hasil daripada dokumentasi.\cite{commerce2014}.

  Menurut Hidayat Rizal, Satrio Adhy, dan Panji Wisnu Wirawan dalam penelitiannya yang berjudul "Perancangan dan Pembuatan \emph{Mobile Learning} Interaktif Berbasis Android Dengan Metode \emph{Personal Extreme Programming}", Penerapan metode \emph{personal extreme programming} memberikan kemudahan dalam memperkirakan kemampuan pengembangan perangkat lunak yang dikembangkan. Hal tersebut ditunjukan pada \emph{burndown chart} yang dihasilkan pada penelitian tersebut.\cite{mlearning}.

  Pada penelitian Indra Faisol Alim dalam "Rancang Bangun Aplikasi Rencana Anggaran Biaya (RAB) Untuk Bangunan Sederhana Di Yogyakarta Berbasis Android Dengan Metode \emph{Extreme Programming}" berhasil mengimplementasikan hasil analisa dari metode \emph{extreme programming} untuk membangun aplikasi tersebut. Dengan hasil aplikasi yang layak menggunakan metode tersebut menurut hasil dari pengujian aplikasi.\cite{rab2016}.

  Dalam penelitian Qoriani Widayati dan Usman Ependi yang berjudul "Rancang Bangun Aplikasi Kamus Istilah Akutansi Pada \emph{Smartphone} dengan Metode \emph{Extreme Programming}". Sasaran dari \emph{extreme programming} adalah tim yang dibentuk berukuran kecil hingga medium, tidak perlu menggunakan tim yang berukuran besar. Hal ini dimaksudkan untuk menghadapi \emph{requirements} yang tidak jelas maupun terjadinya perubahan-perubahan \emph{requirements} yang sangat cepat. Kelebihan yang dimiliki \emph{extreme programming} dibandingkan dengan metode agile yang lain yaitu keperluan berubah dengan cepat, resiko tinggi dan ada proyek dengan tantangan yang baru, tim programmer sedikit, yaitu 2-10 orang, dan mampu mengotomatiskan tes.\cite{kamus-akuntansi2014}.

\section{Landasan Teori}
  \subsection{Software Design Pattern}
    Dua hal yang wajib dipertimbangkan dalam pembuatan perangkat lunak adalah hal-hal yang besifat teknis yang berhubungan dengan arsitektur dan perancangan sistem yang lebih bersifat umum. Permasalahan mengenai kedua hal tersebut selalu ada pada pembuatan perangkat lunak sejak awal perkembangan komputer.

    Tujuan dari merancang kedua hal ini adalah agar perangkat lunak dapat berjalan secara efisien dan mampu menghadapi perubahan-perubahan yang ada.

    Hambatan hambatan yang sering dijumpai dalam pengembangan perangkat lunak dalam hal ini antara lain :
    \begin{enumerate}
      \item Kebutuhan yang akan datang sering tidak bisa dipahami dan merupakan sebuah hal yang sering sekali berubah.
      \item Usaha untuk menggunakan kembali suatu komponen sering kali berujung melibatkan komponen-komponen yang sebenarnya tidak bisa digunakan kembali dan sering kali gagal untuk dilakukan.
      \item Sering sekali terjadi, programmer dan sistem analis kehilangan gambaran secara umum tentang apa yang dimaksud dengan kebutuhan yang akan datang.
      \item Para pengembang cenderung melihat permasalahan ke arah permasalahan kode. Pengembang gagal melihat bahwa penyelesaian masalah melalui perancangan dan pengujian adalah sesuatu yang patut dipertimbangkan dan merupakan jalur yang tepat.
    \end{enumerate}

    Atas dasar permasalah inilah pada proses mengembangkan perangkat lunak aplikasi \emph{mobile} dalam penelitian ini penulis menggunakan salah satu konsep arsitektur perangkat lunak yang disebut dengan Reactive Programming.

    Reactive Programming adalah paradikma pemrograman \emph{asynchronous} yang berkonsentrasi pada aliran data (\emph{data streams}) dan penyebaran perubahan (\emph{propagation of changes}). Contoh, pada arsitektur Model View Controller, Reactive Programming dapat memfasilitasi perubahan dalam model yang otomatis akan diteruskan kepada view yang bersangkutan, dan sebaliknya.

    Arsitektur ini telah tersedia dalam sebuah \emph{framework mobile} bernama React Native yang akan digunakan oleh penulis dalam mengembangkan perangkat lunak \emph{mobile}.

    Dengan menggunakan konsep ini komponen-komponen perangkat lunak lebih terorganisir, sehingga membantu dalam proses pengembangan dan penyesuaian perangkat lunak dengan perubahan yang mungkin terjadi.

  \subsection{Software Development Life Cycle}
    \emph{Software Development Life Cycle} (SDLC) adalah proses pembuatan dan pengubahan sistem serta model dan metodologi yang digunakan untuk mengembangkan sistem-sistem tersebut. Konsep ini umumnya merujuk pada sistem komputer atau informasi. SDLC juga merupakan pola yang diambil untuk mengembangkan sistem perangkat lunak, yang terdiri dari dari tahap- tahap, rencana, analisa, desain, implementasi, uji coba dan pengelolaan.

    Perangkat lunak yang dikembangkan berdasarkan Systems Development Life Cycle akan menghasilkan sistem dengan kualitas yang tinggi, memenuhi harapan penggunanya, tepat dalam waktu dan biaya, bekerja dengan efektif dan efisien dalam infrastruktur teknologi informasi yang ada atau yang direncanakan, serta murah dalam perawatan dan pengembangan lebih lanjut.

  \subsection{Agile Software Development}
    \emph{Agile Software Development} adalah metodologi manajemen pembangunan perangkat lunak yang mempunyai tingkat adaptasi yang tinggi terhadap perubahan yang terjadi di setiap elemen-elemennya. Salah satu ciri dari \emph{Agile Development} adalah adanya proses iterasi yang terus menerus dan evaluasi yang terus berjalan pada setiap proses yang dilewatinnya.

    Beberapa jenis model pengembangan perangkat lunak \emph{agile} antara lain sebagai berikut :
    \begin{enumerate}
      \item Extreme Programming
      \item Feature Driven Development
      \item Lean Software Development
      \item SCRUM
    \end{enumerate}

  \subsection{Extreme Programming}
    \emph{Extreme Programming} adalah sebuah model yang terkenal lincah, menekankan kepuasan pelanggan untuk menciptakan perangkat lunak secara cepat, terampil, dan berkelanjutan. \emph{Extreme programming} mengandung beberapa nilai-nilai sebagai prinsip dasar yaitu \emph{communication}, \emph{simplicity}, \emph{feedback}, dan \emph{courage}.

    Praktek dari model pengembangan \emph{extreme programming} antara lain :
    \begin{enumerate}
      \item Pair Programming
      \item Planning Game
      \item Test Driven Development
      \item Whole Team
    \end{enumerate}

    Pada penelitian ini penulis akan menggunakan metode pengembangan \emph{agile} model \emph{extreme programming} dengan praktek \emph{test driven development}.

  \subsection{Test Driven Development}
    \emph{Test Driven Development} adalah proses pengembangan perangkat lunak yang bergantung pada pengulangan siklus pengembangan yang sangat singkat. Kebutuhan diubah menjadi kasus-kasus pengujian yang sangat spesifik, lalu perangkat lunak dikembangkan hanya untuk lolos uji dari syarat-syarat tersebut. Hal ini bertentangan dengan pengembangan perangkat lunak konvensional yang memungkinkan fitur untuk ditambahkan walaupun terbukti tidak memenuhi persyaratan dari kebutuhan.

    Tidak seperti pengembangan konvensional, dengan model pengembangan \emph{test driven development} memungkinkan proses rencana, analisa, desain, implementasi, dan pengujian dilakukan secara bersamaan. Hal inilah yang menjadikan faktor pengembangan dengan model \emph{test driven development} memiliki kecepatan yang tinggi.

  \subsection{User Stories}
    \emph{User Stories} adalah model yang digunakan untuk melakukan pendataan kebutuhan (\emph{requirements elicitation}) pada metodologi \emph{test driven development}. Fungsinya adalah membuat estimasi waktu untuk \emph{release planning meeting}. Selain itu digunakan untuk mengakomodasi dokumen kebutuhan yang umumnya panjang. \emph{User Stories} ditulis oleh \emph{customer} sebagai sesuatu yang harus dilakukan oleh sistem untuk mereka.

    Berbeda dengan \emph{use case}, dalam \emph{user stories} tidak terdapat detil alur kegiatan dalam suatu stories. Dalam tiap \emph{stories} hanya ditulis apa yang diinginkan oleh \emph{customer} untuk dapat dilakukan oleh sistem yang akan dikembangkan. Deskripsi keseluruhan dari masing-masing \emph{stories} itu sendiri sangat singkat, tidak lebih beberapa baris yang ditulis langsung oleh \emph{customer}.

    Berikut merupakan contoh \emph{user stories} dengan kasus pada sistem \emph{elearning}:
    \begin{enumerate}
      \item Sebagai peserta, saya ingin melihat daftar mata kuliah terbaru yang tersedia sehingga saya bisa mengetahui daftar mata kuliah apa saja yang ditawarkan
      \item Sebagai peserta, saya ingin mencari mata kuliah dengan memasukkan judul atau topik tertentu sehingga saya bisa menemukan kuliah yang mungkin saya minati
      \item Sebagai peserta, saya ingin mendaftarkan diri di suatu mata kuliah tertentu sehingga saya bisa mengikuti pembelajaran di mata kuliah tersebut
    \end{enumerate}

  \subsection{Unit Testing}
    \emph{Unit Testing} merupakan metode yang digunakan untuk melakukan pengujian perangkat lunak dimana masing-masing unit kode sumber, \emph{module}, prosedur penggunaan (\emph{usage procedures}), dan prosedur operasi (\emph{operating procedures}) diuji untuk menentukan apakah sesuai untuk digunakan. \emph{Unit} adalah bagian terkecil yang dapat diuji dari sebuah aplikasi. Dalam pemrograman berorientasi objek, \emph{unit} sering merupakan \emph{class} dan \emph{method}. \emph{Unit Testing} berbentuk sebuah kode pendek yang dibuat oleh pemrogram atau penguji \emph{white box} selama proses pengembangan.

\end{spacing}
% Baris ini digunakan untuk membantu dalam melakukan sitasi
% Karena diapit dengan comment, maka baris ini akan diabaikan
% oleh compiler LaTeX.
\begin{comment}
\bibliography{daftar-pustaka}
\end{comment}

%!TEX root = ../skripsi.tex
%-------------------------------------------------------------------------------
%                            BAB III
%               		METODOLOGI PENELITIAN
%-------------------------------------------------------------------------------
\begin{spacing}{2}
\chapter{METODE PENGEMBANGAN SISTEM}

  Penelitian ini dibuat untuk menganalisis konsep implementasi aplikasi berita berbasis \emph{mobile}, sehingga aplikasi dapat berjalan dengan efektif dan efisien. Sebelum penelitian ini dilakukan riset terlebih dahulu untuk menjaring data serta informasi terkait. Tahap pengumpulan data pada penelitian ini dilakukan dengan metode observasi.

  Kegiatan pengumpulan data secara observasi dilakukan dengan mencoba langsung aplikasi sejenis. Selain itu peneliti juga mengamati fitur dan tata letak dari aplikasi sejenis. Kegiatan observasi dilakukan untuk mengetahui fitur dan keunggulan dari aplikasi sejenis lainnya. Kegiatan observasi dilakukan pada aplikasi \emph{mobile} milik NYTimes, Kompas, dan Detikcom.

  Dalam penelitian ini, metode pengembangan sistem yang digunakan adalah \emph{extreme programming} dengan menggunakan praktek pengembangan \emph{test driven development} dan \emph{tools} UML untuk menggambarkan diagram \emph{use case}. Pemilihan metode ini dikarenakan aplikasi yang dikembangkan berfokus pada \emph{coding} dan \emph{testing} yang mencoba meningkatkan efisiensi dan fleksibilitas dari sebuah proyek pengembangan perangkat lunak. Tahapan metode pengembangan sistem terbagi menjadi 4 tahapan, yaitu \emph{exploration}, \emph{planning}, \emph{iterations}, dan \emph{productionizing}.

  \section{Tahap \emph{Exploration}}
    Pada tahapan ini, terdapat beberapa kegiatan yang dilakukan dalam membangun sebuah aplikasi berita berbasis \emph{mobile}, antara lain:

    \subsection{Identifikasi Ruang Lingkup dan Kebutuhan Sistem}
      Pada tahap ini peneliti mengidentifikasi ruang lingkup dan kebutuhan aplikasi berita berbasis \emph{mobile}, peneliti melakukan observasi terlebih dahulu pada aplikasi sejenis yang sudah ada sebelumnya untuk mengetahui ruang lingkup aplikasi kemudian menganalisis kebutuhan pengguna terhadap aplikasi \emph{mobile} yang akan dibangun. Identifikasi ruang lingkup dan kebutuhan pengguna menghasilkan \emph{user story}.

    \subsection{Menentukan Tools dan Teknologi}
      Pada tahap ini peneliti menentukan \emph{tools} dan teknologi yang dibutuhkan untuk membangun aplikasi berita berbasis \emph{mobile}. Peneliti dapat mengetahui \emph{tools} dan teknologi apa saja yang dibutuhkan berdasarkan hasil identifikasi dan kebutuhan ruang lingkup yang telah dilakukan sebelumnya. \emph{Tools} dan teknologi tersebut berupa bahasa pemgrograman, kerangka kerja, pustaka serta perangkat lunak lainnya yang mendukung pengembangan aplikasi berita berbasis \emph{mobile}.

  \section{Tahap \emph{Planning}}
    Pada tahapan \emph{planning} ada beberapa langkah yang dilakukan dalam membangun sebuah aplikasi berita berbasis \emph{mobile}.

    \subsection{Menentukan Batasan dan Prioritas}
      Pada tahap ini peneliti menentukan batasan dan prioritas berdasarkan identifikasi ruang lingkup yang telah dibuat pada tahap sebelumnya untuk merancang batasan dan prioritas pada masing-masing \emph{user story}. Sehingga ruang lingkup dipecah menjadi beberapa \emph{user story} yang mewakili perangcangan fitur-fitur dari fungsi aplikasi \emph{mobile}.

    \subsection{Membuat Rencana Peluncuran}
      Pada tahap ini peneliti membuat rencana peluncuran untuk merancang ada berapa kali iterasi yang nantinya akan terjadi dalam membangun aplikasi \emph{mobile}. Perancangan iterasi akan terus berulang sesuai dengan kebutuhan \emph{client} yang dituangkan dalam \emph{user story}.

    \subsection{Menyiapkan Uji Penerimaan Pengguna}
      Pada tahap ini peneliti menyiapkan metode apa yang nantinya aplikasi yang dibangun akan di uji, seperti apa proses pengujiannya, dan fitur apa saja yang nantinya akan diuji agar mencapai hasil sesuai dengan kebutuhan aplikasi.

  \section{Tahap \emph{Iterations}}
    Pada tahapan \emph{iterations} dilakukan perulangan selama beberapa kali sesuai dengan lolos atau tidaknya hasil uji untuk mendapatkan hasil yang sesuai dengan keinginan pengguna. Perulangan tersebut dilakukan sesuai dengan \emph{user story} yang telah dibuat pada tahap sebelumnya. Serta tahap pembersihan atau \emph{refactoring} yang berguna untuk memastikan kelayakan dari implementasi. Berikut bagan alur proses dari pengembangan.

    \begin{figure}[H]
      \centering
      \includegraphics[width=1\textwidth]{images/bagan-tdd}
      \caption{Bagan Alur \emph{Test Driven Development}}
      \label{bagan-tdd}
    \end{figure}

    \subsection{Penambahan Tes}
      Pada tahap ini peneliti menulis kode tes baru untuk semua fitur baru yang akan dikembangkan atau menyesuaikan tes sebelumnya. Tes dibuat berdasarkan dari \emph{user story} untuk memenuhi kebutuhan dan menghindari kebutuhan diluar batasan.

    \subsection{Menjalankan Tes Dan Melihat Tes Yang Gagal}
      Pada tahap ini peneliti melakukan pengujian dengan menjalankan semua tes yang telah ditulis, dan melihat tes yang tidak lolos untuk menulis implementasi dari \emph{behavior} yang belum diimplementasikan. Tes yang ditambahkan untuk fitur baru seharusnya tidak lolos pada tahap ini.

    \subsection{Implementasi}
      Pada tahap ini peneliti melakukan implementasi fitur pada aplikasi berdasarkan hasil tes yang tidak lolos uji.

    \subsection{Menjalankan Tes}
      Pada tahap ini peneliti melakukan pengujian setelah implementasi disesuaikan dengan hasil tes sebelumnya. Jika hasil tes lolos uji, maka fitur tersebut sudah sesuai dengan kebutuhan berdasarkan tes yang telah dilewati. Jika hasil tes tidak lolos uji, maka proses akan berulang kembali ke tahap implementasi.

    \subsection{\emph{Refactoring}}
      Pada tahap ini peneliti melakukan pembersihan terhadap kode setelah beberapa kali proses implementasi dan pengujian. Menghapus duplikasi yang terjadi pada implementasi \emph{Object}, \emph{Class}, \emph{Module}, dan \emph{Method} karena harus mempresentasikan tujuan dan penggunaannya, selagi fungsionalitas mereka ditambahkan. Tahap ini berguna untuk mempertajam \emph{readability} dan \emph{maintainability} yang mana dapat menambahkan nilai dalam \emph{lifecyle}.

  \section{Tahap \emph{Productionizing}}
    Pada tahap ini dilakukan pengujian dari aplikasi \emph{mobile}. Pengujian dilakukan dengan cara simulasi aplikasi versi \emph{debug} pada perangkat \emph{mobile} atau \emph{Simulator} perangkat \emph{mobile}.

\section{Kebutuhan Pengembangan Sistem}
	Pengembangan aplikasi berbasis \emph{mobile} membutuhkan perangkat lunak menulis kode aplikasi dan perangkat keras untuk menjalankannya.

	\subsection{Perangkat Keras}
		Spesifikasi perangkat keras yang dibutuhkan untuk mengembangkan aplikasi adalah sebagai berikut:

		\vspace{-0.5cm}

		\begin{enumerate}[a.]
		\itemsep0em
			\item Macbook Air 13"
			\item Memory 4 GB
			\item Harddisk 256 GB
			\item HP Android
			\item iPhone 6
		\end{enumerate}

	\subsection{Perangkat Lunak}
		Perangkat lunak yang dibutuhkan untuk mengembangkan aplikasi adalah sebagai berikut:

		\vspace{-0.5cm}

		\begin{enumerate}[a.]
		\itemsep0em
			\item Sublime Text 3
			\item NodeJS
			\item Reactroton
			\item Web Browser
			\item Git dan Akun Github.com
			\item Akun Travis-CI.com
		\end{enumerate}

\section{Jadwal Penelitian}
	\begin{table}[H]
  \centering
  \caption{Jadwal Penelitian}
  \label{my-label}
  \begin{tabular}{|c|l|l|l|l|l|l|l|l|l|l|l|l|l|}
  \hline
                                & \multicolumn{1}{c|}{}                                    & \multicolumn{12}{c|}{\textbf{Bulan}}                                                                                                                                                             \\ \cline{3-14} 
                                & \multicolumn{1}{c|}{}                                    & \multicolumn{4}{c|}{Oktober}                                                         & \multicolumn{4}{c|}{November}                             & \multicolumn{4}{c|}{Desember}                 \\ \cline{3-14} 
  \multirow{-3}{*}{\textbf{No}} & \multicolumn{1}{c|}{\multirow{-3}{*}{\textbf{Kegiatan}}} & 1                     & 2                             & 3             & 4            & 1        & 2        & 3        & 4                        & 1         & 2         & 3         & 4         \\ \hline
  1                             & Seminar Proposal                                         & \multicolumn{2}{l|}{\cellcolor[HTML]{656565}}         &               &              &          &          &          &                          &           &           &           &           \\ \hline
  2                             & Exploration dan Planning                                 &                       & \multicolumn{2}{l|}{\cellcolor[HTML]{656565}} &              &          &          &          &                          &           &           &           &           \\ \hline
  3                             & Iterations                                               &                       &                               &               & \multicolumn{4}{l|}{\cellcolor[HTML]{656565}} &                          &           &           &           &           \\ \hline
  4                             & Productionizing                                          &                       &                               &               &              &          &          &          & \cellcolor[HTML]{656565} &           &           &           &           \\ \hline
  5                             & Penyelesaian Laporan                                     &                       &                               &               &              &          &          &          &                          & \multicolumn{4}{l|}{\cellcolor[HTML]{656565}} \\ \hline
  \end{tabular}
  \end{table}

\end{spacing}
% Baris ini digunakan untuk membantu dalam melakukan sitasi
% Karena diapit dengan comment, maka baris ini akan diabaikan
% oleh compiler LaTeX.
\begin{comment}
\bibliography{daftar-pustaka}
\end{comment}


%-----------------------------------------------------------------
%Disini akhir masukan Bab
%-----------------------------------------------------------------


%-----------------------------------------------------------------
% Disini awal masukan untuk Daftar Pustaka
% - Daftar pustaka diambil dari file .bib yang ada pada folder ini
%   juga.
% - Untuk memudahkan dalam memanajemen dan menggenerate file .bib
%   gunakan reference manager seperti Mendeley, Zotero, EndNote,
%   dll.
%-----------------------------------------------------------------
\bibliography{IEEEabrv,src/daftar-pustaka}
\addcontentsline{toc}{chapter}{DAFTAR PUSTAKA}
%-----------------------------------------------------------------
%Disini akhir masukan Daftar Pustaka
%-----------------------------------------------------------------

\end{document}