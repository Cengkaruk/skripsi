%!TEX root = ../skripsi.tex
%-------------------------------------------------------------------------------
% 								BAB I
% 							LATAR BELAKANG
%-------------------------------------------------------------------------------
\begin{spacing}{2}
\chapter{LATAR BELAKANG}

\section{Latar Belakang Masalah}
Media massa adalah alat yang digunakan dalam penyampaian pesan-pesan dari sumber kepada khalayak (menerima) dengan menggunakan alat-alat komunikasi mekanis seperti surat kabar, film, radio, TV.\cite{Hafied2005}. Pada perkembangannya, media massa mulai menggunakan teknologi informasi sebagai perantara yang memungkinkan publik berinteraksi lebih cepat dalam mengabarkan berita.\cite{maulana2016}. Ini ditandai dengan munculnya berbagai situs berita milik media massa yang dapat diakses menggunakan perangkat komputer maupun \emph{mobile}.

Kebutuhan mengakses berita melalui media perangkat \emph{mobile} meningkat sesuai dengan hasil riset Pew Research Center yang menyatakan bahwa setengah dari pengguna \emph{smartphone} menggunakan perangkatnya untuk mengakses berita pada tahun 2011.\cite{journalism2012}. Hal ini memperlihatkan bahwa pembaca berita berbasis \emph{mobile} (mobile news) mungkin mengikuti berita secara berkala. Selain itu, pembaca berita berbasis \emph{mobile} memiliki pola penggunaan media dan preferensi berita tertentu.\cite{sylvia2012}. Seperti dengan cara mengakses langsung berita tertentu melalui situs atau aplikasi terkait.\cite{journalism2012}.

Melihat perkembangan tersebut, maka sebuah media massa sudah seharusnya memiliki aplikasi berita \emph{mobile} guna melayani kebutuhan pembaca. Berbagai media massa ternama seperti NYTimes, Wall Street Journal atau media massa lokal seperti Kompas, Detik, dan beberapa media massa lain telah mengembangkan aplikasi berita berbasis \emph{mobile}. Namun pada kenyataannya beberapa aplikasi \emph{mobile} yang dimiliki media massa tersebut sering mengalami masalah dalam penggunaannya, seperti masalah yang sering terjadi pada aplikasi \emph{mobile} milik NYTimes.\cite{martin2011}.

Masalah penggunaan yang dimaksud adalah masalah teknis yang sering terjadi dari aplikasi \emph{mobile}. Penyebab masalah teknis ini sering dilakukan pada proses pengembangan, seperti tidak melakukan proses pengujian yang tepat.\cite{shiv2015}.

Proses pengujian dalam pengembangan tidak lepas dari metode pengembangan yang digunakan. Pada aplikasi berbasis \emph{mobile} metode pengembangan dengan sistem dan pendekatan berbasis proses yang intensif (\emph{proccess-intensive}) berubah menjadi lebih menggunakan pendekatan berbasis \emph{agile} atau proses yang lebih lincah. \emph{Test Driven Development} merupakan salah satu metode pengembangan yang masuk dalam kategori \emph{agile} yang paling sering digunakan.\cite{wasserman2010}.

Dalam studi kasus yang dilakukan oleh IBM Corporation dan North Carolina State University ditemukan bahwa kode sumber yang dikembangkan dengan metode \emph{Test Driven Development} menunjukan selama pengujian verifikasi dan urutan (\emph{regression}), 40\% lebih sedikit mengalami masalah daripada aplikasi yang dikembangkan dengan cara yang lebih tradisional.\cite{laurie2003}

Guna dapat mengembangkan aplikasi \emph{mobile} yang lebih sedikit mengalami masalah, maka penulis mengangkat penelitian ini dengan judul "Rancang Bangun Aplikasi Berita Berbasis Mobile Dengan Metode Test Driven Development".

\section{Rumusan Masalah}
Rumusan permasalahan yang akan diselesaikan dalam penelitian ini yaitu:
\begin{enumerate}
  \item Bagaimana merancang bangun aplikasi berita yang nyaman dan stabil untuk pengguna pada perangkat \emph{mobile}.
  \item Bagaimana menerapkan metode \emph{Test Driven Development} dalam merancang dan membangun aplikasi berbasis \emph{mobile}.
\end{enumerate}

\section{Batasan Masalah}
Batasan masalah yang akan dibahas pada penelitian ini sebagai berikut:
\begin{enumerate}
  \item Penelitian ini fokus pada implementasi metode \emph{test driven development}.
  \item Objek yang dijadikan penelitian adalah aplikasi berita.
  \item Teknologi atau sistem yang digunakan untuk mengelola berita tidak dibahas pada penelitian ini.
  \item Aplikasi ini dibuat menggunakan kerangka kerja React Native.
  \item Penelitian ini tidak membahas mengenai bahasa pemrograman, kerangka kerja, dan pustaka yang digunakan dalam pengembangan.
\end{enumerate}

\section{Tujuan Penelitian}
Sesuai dengan masalah yang telah dirumuskan, maka tujuan dari penelitian ini adalah dapat mengimplementasikan metode \emph{test driven development} pada sistem aplikasi berbasis \emph{mobile} yang dirancang.

\section{Manfaat Penelitian}
Manfaat penelitian yang diharapkan dapat memberikan contoh kepada para pengembang aplikasi mengenai pengembangan aplikasi berbasis \emph{mobile} menggunakan metode \emph{test driven development}.

\end{spacing}
% Baris ini digunakan untuk membantu dalam melakukan sitasi
% Karena diapit dengan comment, maka baris ini akan diabaikan
% oleh compiler LaTeX.
\begin{comment}
\bibliography{daftar-pustaka}
\end{comment}
