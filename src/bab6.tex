%!TEX root = ./template-skripsi.tex

\chapter{HASIL DAN PEMBAHASAN}

\section{Proses Pengembangan Aplikasi}
  Berdasarkan hasil analisis dan pengujian fungsional aplikasi ini, didapat kesimpulan sebagai berikut:

  \subsection{Iterasi Tahap 1}

    \begin{enumerate}
      \item Lorem ipsum is a pseudo-Latin text used in web design, typography, layout, and printing in place of English to emphasise design elements over content. 
      
      \item It's also called placeholder (or filler) text. It's a convenient tool for mock-ups. 
      
      \item It helps to outline the visual elements of a document or presentation, eg typography, font, or layout. Lorem ipsum is mostly a part of a Latin text by the classical author and philospher Cicero.

      \item Its words and letters have been changed by addition or removal, so to deliberately render its content nonsensical; it's not genuine, correct, or comprehensible Latin anymore. 
    \end{enumerate}

  \subsection{Iterasi Tahap 2}
  \subsection{Iterasi Tahap 3}
  \subsection{Productionizing Tahap 1}
  \subsection{Refactoring}
  \subsection{Iterasi Tahap 4}
  \subsection{Iterasi Tahap 5}
  \subsection{Productionizing Tahap 2}
  \subsection{Iterasi Tahap 6}
  \subsection{Productionizing Tahap 3}
  
% Baris ini digunakan untuk membantu dalam melakukan sitasi
% Karena diapit dengan comment, maka baris ini akan diabaikan
% oleh compiler LaTeX.
\begin{comment}
\bibliography{daftar-pustaka}
\end{comment}
