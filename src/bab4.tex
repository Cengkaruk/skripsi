%!TEX root = ./template-skripsi.tex
%-------------------------------------------------------------------------------
%                            BAB IV
%               		ANALISIS DAN PERANCANGAN SISTEM
%-------------------------------------------------------------------------------

\chapter{ANALISIS DAN PERANCANGAN SISTEM}
	\section{Analisis Kebutuhan Sistem}
		Habeo perfecto in sea. Ea deleniti gloriatur pri, paulo mediocrem incorrupte sea ei. Ad mollis scripta per. Incorrupte sadipscing ne mel. Mel ex nonumy malorum epicurei.

		Ne per tota mollis suscipit. Ullum labitur vim ut, ea dicit eleifend dissentias sit. Duis praesent expetenda ne sed. Sit et labitur albucius elaboraret. Ceteros efficiantur mei ad. Hendrerit vulputate democritum est at, quem veniam ne has, mea te malis ignota volumus.

		Eros reprimique vim no. Alii legendos volutpat in sed, sit enim nemore labores no. No odio decore causae has. Vim te falli libris neglegentur, eam in tempor delectus dignissim, nam hinc dictas an.
	
	\section{Perancangan Sistem}		
		Habeo perfecto in sea. Ea deleniti gloriatur pri, paulo mediocrem incorrupte sea ei. Ad mollis scripta per. Incorrupte sadipscing ne mel. Mel ex nonumy malorum epicurei.

		\subsection{Diagram Konteks}
			Consul graeco signiferumque qui id, usu eu summo dicunt voluptatum, nec ne simul perpetua posidonium. Eos ea saepe prodesset signiferumque. No dolore possit est. Mei no justo intellegebat definitiones, vis ferri lorem eripuit ad. Solum tritani scribentur duo ei, his an adipisci intellegat.

		\subsection{User Stories}

	\section{Perancangan Antarmuka Sistem}
		Consul graeco signiferumque qui id, usu eu summo dicunt voluptatum, nec ne simul perpetua posidonium. Eos ea saepe prodesset signiferumque. No dolore possit est. Mei no justo intellegebat definitiones, vis ferri lorem eripuit ad. Solum tritani scribentur duo ei, his an adipisci intellegat.

	\section{Perancangan Peluncuran}
			
			
% Baris ini digunakan untuk membantu dalam melakukan sitasi.
% Karena diapit dengan comment, maka baris ini akan diabaikan
% oleh compiler LaTeX.
\begin{comment}
\bibliography{daftar-pustaka}
\end{comment}