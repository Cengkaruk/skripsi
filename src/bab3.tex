%!TEX root = ../skripsi.tex
%-------------------------------------------------------------------------------
%                            BAB III
%               		METODOLOGI PENELITIAN
%-------------------------------------------------------------------------------
\begin{spacing}{2}
\chapter{METODOLOGI PENGEMBANGAN SISTEM}

\section{Metode Pengumpulan Data}
	Penelitian ini dibuat untuk menganalisis konsep implementasi aplikasi berita berbasis \emph{mobile}, sehingga aplikasi dapat berjalan dengan efektif dan efisien. Sebelum penelitian ini dilakukan riset terlebih dahulu untuk menjaring data serta informasi terkait. Tahap pengumpulan data pada penelitian ini dilakukan dengan metode observasi.

	\subsection{Observasi}
		Kegiatan pengumpulan data secara observasi dilakukan dengan mencoba langsung produk sejenis. Selain itu peneliti juga mengamati fitur dan tata letak dari produk sejenis. Kegiatan observasi dilakukan untuk mengetahui fitur dan keunggulan dari aplikasi sejenis lainnya. Kegiatan observasi dilakukan pada aplikasi \emph{mobile} milik NYTimes, Kompas, dan Detikcom.

\section{Kebutuhan Pengembangan Sistem}
	Pengembangan aplikasi berbasis \emph{mobile} membutuhkan perangkat lunak menulis kode aplikasi dan perangkat keras untuk menjalankannya.

	\subsection{Perangkat Keras}
		Spesifikasi perangkat keras yang dibutuhkan untuk mengembangkan aplikasi adalah sebagai berikut:

		\vspace{-0.5cm}

		\begin{enumerate}[a.]
		\itemsep0em
			\item Macbook Air 13"
			\item Memory 4 GB
			\item Harddisk 256 GB
			\item HP Android
			\item iPhone 6
		\end{enumerate}

	\subsection{Perangkat Lunak}
		Perangkat lunak yang dibutuhkan untuk mengembangkan aplikasi adalah sebagai berikut:

		\vspace{-0.5cm}

		\begin{enumerate}[a.]
		\itemsep0em
			\item Sublime Text 3
			\item NodeJS
			\item Reactroton
			\item Web Browser
			\item Git dan Akun Github.com
			\item Akun Travis-CI.com
		\end{enumerate}

\section{Jadwal Penelitian}
	\begin{table}[H]
  \centering
  \caption{Jadwal Penelitian}
  \label{my-label}
  \begin{tabular}{|c|l|l|l|l|}
  \hline
                                & \multicolumn{1}{c|}{}                                    & \multicolumn{3}{c|}{\textbf{Bulan}}                                            \\ \cline{3-5} 
  \multirow{-2}{*}{\textbf{No}} & \multicolumn{1}{c|}{\multirow{-2}{*}{\textbf{Kegiatan}}} & September                & Oktober                  & November                 \\ \hline
  1                             & Pengajuan Judul                                          & \cellcolor[HTML]{656565} &                          &                          \\ \hline
  2                             & Seminar Proposal                                         &                          & \cellcolor[HTML]{656565} &                          \\ \hline
  3                             & Analisis                                                 &                          & \cellcolor[HTML]{656565} &                          \\ \hline
  4                             & Penyelesaian dan Bimbingan                               &                          & \cellcolor[HTML]{656565} &                          \\ \hline
  5                             & Sidang Skripsi                                           &                          &                          & \cellcolor[HTML]{656565} \\ \hline
  \end{tabular}
  \end{table}
\end{spacing}
% Baris ini digunakan untuk membantu dalam melakukan sitasi
% Karena diapit dengan comment, maka baris ini akan diabaikan
% oleh compiler LaTeX.
\begin{comment}
\bibliography{daftar-pustaka}
\end{comment}
